\chapter{Estructura y Manejo de datos}
Saber la estructura y como manejar datos nos permitira controlar el flujo de la ejecucuion basado en condiciones o ejecutar un codigo de manera repetitiva. Manejo de datos envuelve operaciones sobre los datos como crear, recivir manipular y almacenar atravez de tu programa.
\\
Controlar estructuras incluye
\begin{itemize}
  \item Condicionales (if , else, switch)
  \item Loops (for, while, do-while)
\end{itemize}
Manejo de datos
\begin{itemize}
  \item Manipular strings, arrays y objetos
  \item Verificar tipos de datos y convertirlos entre ellos
\end{itemize}

\section{Condicionales}
Para el manerjo de concdicionales tenemos las siguientes opciones
\\
La manera mas sencilla y facil

\begin{lstlisting}[language=JavaScript,caption={Ejemplo if else basico}]
let x = 1;
let y = 2;
if(x < y){
    console.log("x es menor que y");
} else {
    console.log("y es menor que x");
}
\end{lstlisting}
Ahora vemos un if en una sola linea

\begin{lstlisting}[language=JavaScript,caption={Ejemplo if una linea}]
let x = 1;
let y = 2;
if (x < y) console.log("x es menor que y");
\end{lstlisting}

\begin{lstlisting}[language=JavaScript,caption={Ejemplo if else una linea}]
let x = 1;
let y = 2;
if (x < y) console.log("x es menor que y"); else console.log("y es menor que x");
\end{lstlisting}
\subsection{Operador Ternario}
El operador ternario es una forma concisa de escribir una expresión condicional en JavaScript (y en otros lenguajes de programación). Se utiliza para evaluar una condición y devolver uno de dos valores basados en el resultado de esa evaluación. \cite{OpenAI2023}
\\

condición ? valorSiVerdadero : valorSiFalso;

\begin{lstlisting}[language=JavaScript,caption={Ejemplo if else una linea}]
let x = 1;
let y = 2;
let respuesta = (x < y) ? "x es menor que y": "y es menor que x";
\end{lstlisting}
Ejemplo con varios tenarios

\begin{lstlisting}[language=JavaScript,caption={Ejemplo if else elif ternarios}]
let edad = 32;
let respuesta = (0 < edad < 2) ? "Bebe":
                (edad < 12) ? "Nino":
                (edad < 18) ? "Adolecente":
                (edad < 70) ? "Adulto":
                (edad < 110) ? "Adulto mayor":
                "La edad no Existe";
\end{lstlisting}
Con esta explicacion ahora haremos un ejemplo con un archivo HTML
\lstinputlisting[language=javascript,linerange={1-8, 11-12},caption=ejemplo.html]{/chapter_02/ejemplo.html}
\lstinputlisting[language=javascript,caption=ejemplo\_01.js]{/chapter_02/ejemplo_01.js}
Si abres el archivo ejemplo\_01.html con un navegador podras ver apretando F12 un mensaje en la consola
\section{Iteraciones o Loops}
Ejemplo de como hacer un loop que es basicamente un codigo que ejecuta una accion multiples veces como el usuario requiera.
\lstinputlisting[language=JavaScript,caption=ejemplo\_02.js]{/chapter_02/ejemplo_02.js}
Ejemplo en una sola linea

\begin{lstlisting}[language=JavaScript,caption={Ejemplo for una linea}]
for (let i = 1; i <= 10; i ++) console.log(i);
\end{lstlisting}
\section{Verificacion de tipos de datos}
\begin{lstlisting}[language=JavaScript,caption={Ejemplo tipos de datos}]
let name = Jon;
let edad = 32;
let esEstudiante = false;
console.log(typeof name); // "string"
console.log(typeof edad); // "number"
console.log(typeof name); // "boolean"
\end{lstlisting}

\lstinputlisting[language=javascript,caption=ejemplo\_03.html,linerange={1-11}]{/chapter_02/ejemplo_03.js}

{\textbf {\huge Practica}}
\\
Dado una entrada escribe si es un numero, si es un numero debuelve un mensaje en la consola si este es positivo o negativo. agrega el script a ejemplo.html

\lstinputlisting[language=javascript,caption=practica\_01.js,linerange={1-100}]{/chapter_02/practica_01.js}

%\lstinputlisting[language=javascript,caption=ejemplo.js,linerange={1-11}]{/chapter_02/ejemplo_01.js}
%
%\begin{lstlisting}[language=JavaScript,caption={Ejemplo if else una linea}]
%let x = 1;
%let y = 2;
%if (x < y) console.log("x es menor que y"); else console.log("y es menor que x");
%\end{lstlisting}
%
