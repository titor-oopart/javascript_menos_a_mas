\chapter{Introduccion}
JavaScript es un lenguaje ampliamente usado en la Web para poder trabajar con JavaScript es necesario que tengas conocimiento de \href{https://developer.mozilla.org/en-US/docs/Glossary/World_Wide_Web}{World Wide Web} para poder entender como funciona el internet y sus protocolos y tambien debes tener conocimiento de \href{https://developer.mozilla.org/es/docs/Glossary/HTML}{HTML}
\\

JavaScript es usado en todos los navegadores web del mundo y hace que las paginas web dejen de ser estaticas usando solo HTML y pasen a ser dinamicas pudiendo conectarse a base de datos hacer requerimientos a API (Interfaz de programación de aplicaciones)
\\

Este libro esta siendo escrito para poder aprender JavaScript y empezar a desarrollar aplicaciones y soluciones a problemas que el usuario vea conveniente hacerlo, para esto se enfocara en la parte basica de los primeros pasos en JavaScript teniendo en cuenta que el usuario ya tiene experiencia en algun lenguaje previo y tiene conocimiento sobre variables funciones e iteraciones.

%Small desctiption \cite{lehman2006biblatex} en la figura \ref{figura1}

%\begin{lstlisting}[language=python,caption={Python example}]
%def test(name):
%	print(name)
%\end{lstlisting}
%\begin{table}[h]
%	\centering
%	\begin{tabular}{l | l | l}
%		A & B & C \\
%		\hline
%		1 & 2 & 3 \\
%		4 & 5 & 6
%	\end{tabular}
%	\caption{very basic table}
%	\label{tab:abc}
%\end{table}
%
%\begin{table}[h]
%	\centering
%	\begin{tabular}{| l c r |}
%		\hline
%		1 & 2 & 3 \\
%		4 & 5 & 6 \\
%		7 & 8 & 9 \\
%		\hline
%	\end{tabular}
%	\caption{A simple table}
%\end{table}
%
%
%\begin{figure}[h]
%
%	\centering
%
%	\includegraphics[width=0.5\textwidth]{image/cover.jpg}
%
%	\caption{Figure description.}
%
%	\label{figura1}
%
%\end{figure}
